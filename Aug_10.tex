\documentclass[hidelinks]{article}
\usepackage[a4paper, total={7in, 10in}]{geometry}
\usepackage[dvipsnames]{xcolor}
\usepackage{amsmath}
\usepackage{tikz}
\usepackage{tkz-euclide}
\usepackage[unicode]{hyperref}
\usepackage[all]{hypcap}
\usepackage{fancyhdr}

\usetikzlibrary{angles,calc, decorations.pathreplacing}

\definecolor{carminered}{rgb}{1.0, 0.0, 0.22}
\definecolor{capri}{rgb}{0.0, 0.75, 1.0}
\definecolor{brightlavender}{rgb}{0.75, 0.58, 0.89}

\title{\textbf{Compiler Design}}
\author{Sahil Muhammed}
\date{August 10th, 2025}
\begin{document}
\hypersetup{bookmarksnumbered=true,}
\pagecolor{black}
\color{white}
\maketitle

\begin{Large}
\tableofcontents
\end{Large}%
\pagebreak

\section{Compiler and Interpreter}

\begin{itemize}
    \item A compiler is a software or program that translates programs from a source language to a target language so that both programs are equivalent during run-time.
    \item An interpreter is a language processor that directly executes the instructions from the source language and gives the output directly, skipping the translation process in comparison to a compiler.

    \subsection{Analogy — Language translation}

    \begin{itemize}
        \item Imagine you write a book in French and your audience speaks English.

        \item \textbf{Compiler approach:}

        You hire a translator who translates the entire book into English, prints it, and hands it to your audience. \\ 

        Once done, the audience can read it anytime without the translator. \\

        \textbf{Drawback}: takes time before anyone can start reading.

        \item \textbf{Interpreter approach:}

        You hire a translator to stand next to the audience and read your French book aloud, translating sentence-by-sentence in real time. \\

        People can start hearing the story right away. \\

        \textbf{Drawback}: the translator must be there every time, and real-time translation is often slower.
    \end{itemize}

    \item There are hybrid use-cases of both compilers and interpreters. For example, Java compiles the program into bytecode and then the intermediate program is then interpreted (line-by-line by execution). These bytecodes are interpreted by a virtual machine. Its application being that it can be compiled in one computer and can then be interpreted across a network.

    \item There are some Java compilers called "Just-In-Time" (JIT) compilers which try to improve the interpretation of bytecodes by directly translating portions of a program which occur more frequently than other parts (also called \textit{hotspots}) into native machine code to save time when its run the next time.

    \item The name comes from considering the phase at which it compiles the program. A compiler compiles the entire program into assembly (or some low-level language) before being executed. In this case, the compilation takes place during the interpretation and its only done for chunks which are repetitive.

\end{itemize}
% \section{Question 1}
% Question text

% {\color{SkyBlue}
% Answer work

% \color{Thistle}{\textbf{Answer}}
% }

% \begin{tikzpicture}

% % coords for drawing angles
% \coordinate (origin) at (0,0);
% \coordinate (negx) at (-5,0);
% \coordinate (posx) at (5,0);
% \coordinate (negy) at (0, -5);
% \coordinate (vecb) at (2.4075, -9.1945);
% \coordinate (veca) at (-2.4075, 3.1945);

% %axes
% \draw[<->,ultra thick] (-5,0) coordinate (A) --(5,0) node[right]{$x$};
% \draw[<->,ultra thick] (0,-5)--(0,5) node[above]{$y$};

% %vec a
% \draw[->, thick, capri] (0,0) coordinate (B) -- (-2.4075, 3.1945) coordinate (C) node[left]{\textbf{A}};

% % angle for vec a
% \draw pic[thick, capri, "53$^\circ$", draw, <-,angle radius=1cm,angle eccentricity=1.4] {angle = veca--origin--negx};

% % opposite angle for vec a
% \draw pic[thick, "127$^\circ$", draw, ->,angle radius=1cm, angle eccentricity=1.4] {angle = posx--origin--veca};

% % magnitude label for vec a
% \draw [thick, capri, decorate,decoration={brace,amplitude=6pt,mirror,raise=1ex}] (0,0) -- (-2.4075, 3.1945) node[midway, yshift=2em, xshift=2em]{8.0 m};

% % vec b
% \draw[->, thick, carminered] (0,0) coordinate (B) -- (2.4075, -9.1945) coordinate (C) node[left]{\textbf{B}};

% % angle for vec b
% \draw pic[thick, carminered, "14.7$^\circ$", draw, ->,angle radius=5cm, angle eccentricity=1.1] {angle = negy--origin--vecb};

% % label for vec b
% \draw [thick, carminered, decorate,decoration={brace,amplitude=6pt,raise=1ex}] (0,0) -- (2.4075, -9.1945) node[midway, xshift=2.5em, yshift=0.5em]{19.0 m};

% %vec c
% \draw[->, thick, brightlavender] (0,0) coordinate (B) -- (0, -6) coordinate (C) node[below]{\textbf{A + B}};

% % label for vec c
% \draw [thick, brightlavender, decorate,decoration={brace,amplitude=6pt,raise=1ex, mirror}] (0,0) -- (0, -6) node[midway, xshift=-3em]{12.0 m};
% \end{tikzpicture}



\end{document}

